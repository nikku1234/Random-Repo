\documentclass[a4paper,12pt]{article}

\begin{document}

\title{My First Document}
\author{Nikhil Ramesh}
\date{\today}
\maketitle



% This code starts a table:  \begin{tabular}{...}    Where the dots between the curly brackets are replaced by code defining thecolumns: l for a column of left-aligned text (letter el,notnumber one). r for a column of right-aligned text., c for a column of centre-aligned text.
 %     |for a vertical line.For example,{lll}(i.e.  left left left) will produce 3 columns of left-alignedtext with no vertical lines , while{|l|l|r|}(i.e.|left|left|right|) will produce 3 columns — the first 2 are left-aligned, the third is right-aligned, and there are vertical lines around each column.The table data follows the \begin command:
 %     & is placed between columns.
 %     \\is placed at the end of a row (to start a new one)
%     \hlineinserts a horizontal line.  
%     \cline{1-2}inserts  a  partial  horizontal  line  between  column  1  andcolumn 2.
%The command\end{tabular}finishes the table.


\begin{tabular}{|l|l|}
Apples & Green 
\\Strawberries & Red 
\\Oranges & Orange 
\\\end{tabular}

\newpage

\begin{tabular}{rc}
Apples & Green \\
\hline
Strawberries & Red \\
\cline{1-1}
Oranges & Orange \\
\end{tabular}

\newpage

\begin{tabular}{|r|l|}
\hline
8 & here’s \\
\cline{2-2}
86 & stuff \\
\hline \hline
2008 & now \\
\hline
\end{tabular}


\newpage





\end{document}